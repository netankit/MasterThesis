\chapter{Conclusion}\label{chapter:conclusion}
Sentiment and Emotions is what makes us human; it separates us from artificial machines. As an AI researcher and a computer scientist, through this thesis, the task was to bridge this gap between humans and machines a little closer, by understanding in a natural way using algorithms and natural language processing the sentiments and emotions exhibited by humans in raw subjective data (text or speech). 
\newline

 At large, the primary focus was to learn in depth about the deep learning methods for sentiment classification of textual data. We started out with a brief intro to conventional methods for Sentiment Analysis and then we presented several deep learning approaches, which are current state of the art in both language modelling and sentiment classification. Some of the word representations, we discussed in dept were: Bag of Words, Word2Vec, GloVe, Polyglot and Paragraph Vectors. We also learned about the theory and principles of working behind a neural network and largely how they fit in the realms of deep learning and facilitate unsupervised learning and how they are able to automatically learn intricate features from raw un-labeled data.  
 
 A comprehensive study involving concatenation of word vectors and building sentence vectors from those concatenated representation and then applying the same to sentiment classification is also presented.  Additionally, we tried to gravitate around from text to speech data and evaluated conventional GMM and UBM methods for emotion recognition in speakers speech data in noisy environments. To keep up with the main theme of this thesis, a deep neural network based method for emotion recognition was also mentioned briefly.
\newline

A good extension to this study is to naturally blend in the methods for Deep Learning with that of Language processing, as already done in the RNTN models and the Tree-LSTM models which try to capture the syntax of the sentence and also tries to capture long term relationship between phrases, and bring forward a unified learning algorithm which can capture scope of negation better. Also, one can also think about methods which involve capturing sarcasm in review based entities. This is indeed a very tough problem even for humans and developing algorithms for the same will be a challenge in itself. 